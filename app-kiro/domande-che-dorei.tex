% Created 2023-07-23 Sun 20:01
% Intended LaTeX compiler: pdflatex
\documentclass[11pt]{article}
\usepackage[utf8]{inputenc}
\usepackage[T1]{fontenc}
\usepackage{graphicx}
\usepackage{longtable}
\usepackage{wrapfig}
\usepackage{rotating}
\usepackage[normalem]{ulem}
\usepackage{amsmath}
\usepackage{amssymb}
\usepackage{capt-of}
\usepackage{hyperref}
\author{Biggie Dickus}
\date{\today}
\title{}
\hypersetup{
 pdfauthor={Biggie Dickus},
 pdftitle={},
 pdfkeywords={},
 pdfsubject={},
 pdfcreator={Emacs 28.2 (Org mode 9.5.5)}, 
 pdflang={English}}
\begin{document}

\tableofcontents

\section{Clos and company}
\label{sec:org86bef48}
\begin{itemize}
\item T-S e condizioni di blocco (due volte)
\item formula di clos
\item formula di lee (due volte)
\item ricavare la formula di lee per il caso S-S-S
\item formula di clos per strutture T-S-T
\end{itemize}

\section{Reti}
\label{sec:org022ab86}
\subsection{WiMax}
\label{sec:org13e8d85}
\subsubsection{Classi di servizio}
\label{sec:orgfdac13a}
\subsection{Lan}
\label{sec:orgffd1b31}
\subsubsection{Dispositivi per la connessione}
\label{sec:orgc2ac271}

\section{Cazzo è un}
\label{sec:org3d7d984}
\subsection{SDH}
\label{sec:org8d8c72d}
\subsection{SSN7}
\label{sec:org78a3fa0}
\subsection{gossiping}
\label{sec:orgf2f9a4f}
\subsection{FDDI}
\label{sec:org771105f}
\begin{itemize}
\item \textbf{Fiber Distributed Data Interface}
\item \textbf{Fiber Distributed Data Interface}
\item \textbf{Fiber Distributed Data Interface}
\end{itemize}
protocollo progettato che usa la fibra come mezzo trasmissivo
la fibra ottica trasmette \(100 Mbps\) l'estensione massima di una tale rete è tanti tanti kilometri e permette di collegare sulle 500 stazioni

standardizzata ansi

evoluzione del token ring

\begin{itemize}
\item usa solo fibra
\item area di lavoro estesa (MAN)
\item resilienza ai guasti sia dei collegamenti che dei nodi
\end{itemize}

\subsubsection{Topologia}
\label{sec:org7243735}
doppio anello, un anello va da una parte, l'altro anello va dall'altra parte
le comuncazionu usano uno solo dei due anelli, detto \emph{primario} (di solito quello esterno)
l'altro anello, quello \emph{secondario}, viene attivato in caso di guasti o interruzioni dell'anello primario, aumentando la resistenza e flessibilità della rete.

in casi particolari l'anello secondario può anche essere utilizzato per mandare ancora più roba portando il la capacità nominale di canale a \(200 Mbps\)

\subsubsection{Utilizzi}
\label{sec:org03cc5b9}
la rete fddi è molto resistente e veloce, rendendola adatta a impieghi quale rete di backbone (o dorsali per lan) o per il trasferimento di molti dati in fretta e in modo accurato per ambiti sensibili quali ambiti ospedalieri.

\subsubsection{A livello MAC}
\label{sec:org049584f}
l'accesso al canale da parte dei nodi avviene mediante tecnica ad accesso ordinato che si richiama al token ring
supporta la trasmissione di traffico sincrono e asincrono
per la trasmissione isocrono è stato definito uno standard \texttt{FDDI-II}, più contorto e meno cagato dell'\texttt{FDDI}.

Il traffico sincrono non prevede livelli di priorità, a differenza del traffoco asincrono, per il quale ci sono 8 livelli di priorità.
al traffoco sincrono viene sempre garantito il diritto di trasmissione mentre per il traffico asincrono si utilizza una modalità "best effort" (si trasmette quando possibile)

la possibilità di gestire traffico sincrono e asicrono comporta la necessità di avere dei buffer in cui memorizzare sto traffico

\subsubsection{Tipi di traffico}
\label{sec:org85e533f}
\begin{description}
\item[{token frame}] (di segnalazione) ha una struttura predefinita nota a tutti i nodi, utilizzato per la condivisione dell'accesso, la stazione che ha il token frame è quella abilitata alla trasmissione, una volta terminato l'invio dei dati rilascia il token frame, mandandolo dopo aver mandato i dati
\item[{data frame}] frame con dati
\end{description}

\subsubsection{Como il traffico}
\label{sec:org83e70b0}
c'è un check di integrità
quando il destinatario si riconosce nel messaggio lo segna come letto
tutti i nodi ripetono tutti i messaggi inviati tranne quelli mandati da loro stesso
solo il nodo creatore/mittente di un messaggio può eliminare il messaggio in questione

\subsubsection{Sborra token}
\label{sec:org0b14d37}
\begin{description}
\item[{token target rotation time}] \[ TTRT \geq \sum_{i = 1}^{N_d} \alpha _i + \sum_{i=1}^{N_d} d_i \]
dove
\begin{description}
\item[{\(\alpha _i\)}] è il tempo necessario alla stazione i-esima per la trasmissione del traffico sincrono
\item[{\(d_i\)}] è il tempo di passaggio del token da una stazione alla successiva, detto \emph{walking time}
\end{description}
\end{description}

per come è definito il \texttt{TTRT} indica il tempo di riferimento necessario al fine di assegnare a tutti i nodi della rete il diritto di trasmissione congruente con le necessità di accesso dichiarate
una volta determinato e condviso il valore del \texttt{TTRT} allora la gestione dell'accesso viene implementata in maniera distribuita ed indipendete
quindi in ogni nodo è reso disponibile un contatore che misura il \texttt{Token Rotaton Time} (o \texttt{TRT}) inteso come il tempo effettiv che incorre tra una ricezione del token e quella dopo

ogni nodo tiene inoltre anche un contatore a solo decremento che definisce il tempo di accesso del nodo alla rete (quindi il tempo per cui ha tenuto il token, il contatore si chiama \texttt{Token Holding Time}, o \texttt{THT}.

\[ THT := \begin{cases} \alpha _i & \text{ se } TTRT - TRT \leq 0 \\ TTRT-TRT & \text{ altrimenti } \end{cases} \]



\subsection{dqdb}
\label{sec:orgf75642a}
\subsection{ATM}
\label{sec:org3be1d3a}
\subsection{PCF e DCF}
\label{sec:orga094067}

\section{Poi}
\label{sec:org5a42899}
\subsection{TCP IP}
\label{sec:org3711004}
\subsection{Terminale nascosto, terminale esposto}
\label{sec:org8ecf2b3}
\end{document}