% Created 2023-07-25 Tue 10:58
% Intended LaTeX compiler: pdflatex
\documentclass[11pt]{article}
\usepackage[utf8]{inputenc}
\usepackage[T1]{fontenc}
\usepackage{graphicx}
\usepackage{longtable}
\usepackage{wrapfig}
\usepackage{rotating}
\usepackage[normalem]{ulem}
\usepackage{amsmath}
\usepackage{amssymb}
\usepackage{capt-of}
\usepackage{hyperref}
\author{Biggie Dickus}
\date{\today}
\title{}
\hypersetup{
 pdfauthor={Biggie Dickus},
 pdftitle={},
 pdfkeywords={},
 pdfsubject={},
 pdfcreator={Emacs 28.2 (Org mode 9.5.5)}, 
 pdflang={English}}
\begin{document}

\tableofcontents

\url{https://en.wikipedia.org/wiki/Wireless\_sensor\_network}

\section{Data Centric Forwarding}
\label{sec:orgd6b068e}
instradamento \texttt{Data Centric} (\texttt{DC})
tecnica di instradamento per wsn alternativa all'address based

evitano problemi dell'address based quali
\begin{itemize}
\item dover associare un indirizzo a ogni dispositivo della rete
\item prevedere la mobilità dei dispositivi connessi alla rete
\item prevedere un aggiornamenoto periodico dei possibili cammmini interni alle wsn
\end{itemize}

invece di considerare il destinatario del pacco ne analizzano il significato, cazzo è questo, cazzo vuole dire, e su cosa.
poi vede dove mandarlo

permette le seguenti funzionalità
\begin{description}
\item[{data dissemination}] diffondere nell'intera wsn l'informazione acquisita dai sensori
\item[{network centric processing}] elaborazione dei dati, spesso aggregazione, per evitare ridondanze sulla rete, processing \emph{on edge}
\end{description}

per evitare di mandare troppe merdate si usano indicatori di posizione atti aidentificare l'area di raccolta,richiesta, o interesse delle stesse
si usano anche indicatori temporali per capire di quando è il dato, oltre che per elaborarli meglio, anche per poter scartare dati vecchi, che non mi servono.

\subsection{Flooding}
\label{sec:org3b94a8d}
\begin{itemize}
\item lo mandi a tutti i vicini
\item per evitare di mandarlo all'infinito usi un time to live
\item bello il time to live ma poi devi anche implementare i meccanismi per supportarlo
\end{itemize}

\subsection{Gossiping}
\label{sec:orgc641552}
variante del flooding che, per evitare di allagare la rete, non manda il messaggio a tutti i vicini ma a un solo vicino su basi probabilistiche

varianti del gossiping provano a ottimizzare la scelta o a utilizzare informazioni aggiuntive come , ad esempio sui nodi vicni (smart gossiping)

\section{Direct Diffusion}
\label{sec:org7a5f1f1}
identifica i dati generati da un sensore usando una coppia atttributo-valore
un qualsiasi altro membro della rete esprime il suo interesse verso dati mandando una lista di coppie attributo valore

\subsection{Fasi}
\label{sec:orgc5b6af2}
la tecnica prevede le seguenti fasi
\begin{enumerate}
\item \textbf{interest e gradient} : un elemento della \texttt{WSN} interessato (di solito un sink) manda in flooding la sua richiesta di interesse.
Il passaggio serve solo a scoprire chi ha quell'informazione.
a ogni rotta prorietario/interessato viene associata una misura, detta \emph{gradiente}, circa la qualità della stessa (bitrate, affidabilità, etc), si scelgolo una o poche tra le rotte migliori
\item \textbf{data propagation} : una volta ricevuta la dichiarazione di interesse e verificato che questa è pertinente a quanto sta effettivamente misurando, il nodo diventa sorgente e inoltra l'informazione verso l'interessato tramite le rotte trovate prima.
se il messaggio arriva ad un nodo non interessato, questo se ne fotte e non lo inoltra.
\item \textbf{reinforcement} : il sink può procede a reinforzare la richiesta aggiungendo ulteriori specifiche per, ad esempio, eliminare percorsi poco affidabili o con bit rate basso, si tengono così solo i percorsi migliori e si sforza meno la rete.
\end{enumerate}

direct diffusion opera in modalità reattiva query driven, quindi non si adatta ad applicazioni che richiedono un invio costante di dati verso il sink    

\section{SPIN}
\label{sec:org913ff96}
\begin{quote}
Sensor Protocol for Information via Negotion
\end{quote}

si basa sulla diffiusone nella wsn non delle informazioni ma di metadati
prima che avvenga la trasmissione dell'info il nodo manda un \texttt{ADV}\footnote{advertisement} che "pubblicizza" i metadati dei dati posseduti
i nodi vicini interessati rispondono con una richiesta di invio (\texttt{REQ}), per poi ricevere i dati (\texttt{DATA})

i nuovi proprietarii dell'informazione procederanno quindi al diffondere il verbo dei dati che hanno appena acquisito, e gli interssati che lo riceverrano andranno avanti così fino a scopirire che mo' non interessa a nessuno :(

\subsection{Problema}
\label{sec:orgbc8a4e5}
se ho un nodo interessato, che non ha manco un vicino che gle frega, quel nodo non avrà mai il modo di ricevere il verbo

ma migliora comunque le prestazioni rispetto al flooding, che tra un po' è esponenziale sul time to live, quindi va bene.
\end{document}