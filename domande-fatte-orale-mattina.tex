% Created 2023-07-24 Mon 12:01
% Intended LaTeX compiler: pdflatex
\documentclass[11pt]{article}
\usepackage[utf8]{inputenc}
\usepackage[T1]{fontenc}
\usepackage{graphicx}
\usepackage{longtable}
\usepackage{wrapfig}
\usepackage{rotating}
\usepackage[normalem]{ulem}
\usepackage{amsmath}
\usepackage{amssymb}
\usepackage{capt-of}
\usepackage{hyperref}
\author{Biggie Dickus}
\date{\today}
\title{}
\hypersetup{
 pdfauthor={Biggie Dickus},
 pdftitle={},
 pdfkeywords={},
 pdfsubject={},
 pdfcreator={Emacs 28.2 (Org mode 9.5.5)}, 
 pdflang={English}}
\begin{document}

\tableofcontents

(ho provato a segnare anche le risposte date, ma provare ad ascoltare cosa dicono e scriverlo contemporaneamente è più difficile del previsto, quindi quanto riportato presenta buchi e male interpolazioni, visto che non so abbastanza sulla materia da riempire i buchi non compresi)
(mi scuso per le battutine del cazzo)
(e per il pdf formattato strano con le righe e la puteggiatura)
\section{N}
\label{sec:org9e41c15}
\subsection{(roba che non ho segnato)}
\label{sec:org4b1b278}
\subsection{Ha chiesto l'FDDI da qualche parte}
\label{sec:org09df73d}
\subsection{S-S-S}
\label{sec:orgaab8ff2}
\begin{description}
\item[{struttura S-S-S}] 

\item[{analisi di clos per le strutture S-S-S}] (si basa sullo studio del caso peggiore, si assume che \ldots{} occupati e occupati)
\item[{dhcp}] \begin{itemize}
\item cos'è
\item a cosa serve
\item[{come funziona}] il client manda al server la richiesta\ldots{}
il server che riceve la richiesta da al client l'indirizzo ip disponibile
il client sceglie tra gli indirizzi ip quello più vantaggioso in base a varii criteri
comunica la scelta dell'indirizzo al server
prima che tale indirizzo appartenga al client è necessaria una conferma finale da parte del server visto che mentre il client sceglie può essere che l'ip l'hanno preso altri
\end{itemize}
\end{description}

\subsection{Come si fa a distinguere una sequenza da un'altra}
\label{sec:orgbc20248}
(numero di sequenza)

\subsection{PLS}
\label{sec:org81e8632}
una tacnica di routing che si basa su\ldots{}
si associa a più \ldots{} una stessa etichetta
l'etichetta è associata al tipo di traffico e \ldots{} indirettamente al dispositivo

\subsection{Differenza tra routing \ldots{} e routing PLS}
\label{sec:org93a2f6d}
qualcosa proprio a livello formale che nel routing classico non è proprio possibile
legata a come il PLS lavora, una particolarità che consente di implementare \ldots{}

l'abbinamento sorgente destinazione \ldots{} è uno solo, nell'NPLS no apputno perchè \ldots{}

ha preso 28

\section{N + 1}
\label{sec:org537d5d8}
\subsection{Reti DQDB}
\label{sec:orgd3cf774}
le reti dqdb (e disegna l'affare con due bus e tre contatori sulla lavagna)
quando un nodo \ldots{} tutti gli altri nodi vengono \ldots{} attraverso \ldots{} un bus e l'altro bus \ldots{}

\subsubsection{mi fai vedere la struttura interna di un nodo?}
\label{sec:org50f3b31}
e come \ldots{}

mi fai però lo schemettino che \ldots{}
ok il traffico però l'architettura interna al nodo per gestirlo
\begin{description}
\item[{contatore add drop}] incrementa se p = 1, decrementa se p = 0, conta quanti nodi si sono già prenotati, quando un nodo vuole trasmettere il numero viene passato al drop, quindi è conensso al contatore drop
\item[{contatore drop}] decrementa se s = 0 e conta quanti nodi hanno priorità rispetto a \ldots{}
\item[{buffer}] 
\end{description}

durante il periodo di attesa questo qui che fa, si ferma oppure va a \ldots{}   
io prenoto e trasferisco, in attesa che \ldots{}

la domanda è semplicemente, durante il periodo di attesa, il contatore drop?
continua a funzionare perchè si deve rispettare l'ordinamento temporale per rispettare la fila d'attesa

\subsubsection{Inconveniente}
\label{sec:orgee7b4b5}
è simpatica, è bellina, però ha un'incoveniente, apparentemente sembra ideale, rispetti sempre la priorità\ldots{}
però purtroppo c'è un problema, che sono privilegiati i nodi più vicini a\ldots{}

\subsection{Il bridge}
\label{sec:orgcbcc51a}
\begin{itemize}
\item che cos'è
\item come funziona
\item cosa si intende con politica di autoapprendimento
\end{itemize}

\subsubsection{autoapprendimento}
\label{sec:orgeb0a817}
se l'indirizzo che arriva \ldots{} non è già presente nella tabella questa viene aggiornata
salva l'indirizzo su un buffer

il bridge te lo metti e fa da se
all'inizio la memoria è tutta vuota

te nel pacchetto hai un campo sorgente e un campo destinazione, il bridge che fa? non conosce mica la rete, usa il campo sorgente perchè sai che lo ricevi da una porta e \ldots{}

\subsection{Tecnica spin}
\label{sec:org70444d3}
\begin{itemize}
\item cos'è
\item dove si usa
\item \ldots{}
\end{itemize}

è una tecnica usata per routing datacenter basata su metadati, questa non è un'informazione completa è per l'apputno una descrizione
uno manda\ldots{} e pubblicizza il proprio metadato, quelli interessati mandano una richiesta e \ldots{} prenotano

quello manda in broadcast e \ldots{}

\subsubsection{Qual'è il problema di questa tecnica?}
\label{sec:org1149644}
sembra proprio bellina però ha un inconveniente da sistemare
non tutti i nodi vengono raggiunti dalla pubblicizzazoine, quindi?
la conseguenza di questo? che può succedere? Che se nessuno di quelli vicni è interessato allora \ldots{}

questo ha preso 27, è andato bene

\section{N + 3}
\label{sec:orgf4ae6d1}
(stampelle, potrebbe essersi limitato apposta da domande che richiedessero la lavagna)
\subsection{Ci dimostri la formula di lee}
\label{sec:orga0f925a}
rispetto all'analisi di Clos vede il blocco come un evento aleatorio
in questo caso una probabilità di blocco bassa viene accettata, accetta il fatto che possa esserci una condizione di blocco

si chiama \(\alpha\) la probabilità con cui una rete può essere utilizzata \ldots{} \(\frac{1}{k}\) \ldots{} \(n\) dove \(n\) sarà il numero di linee in ingresso

la probabilità che un'uscita boh sia occupata sarà \(\frac{n \times \alpha}{k}\)
questa analisi non è però precisa, se si inserisce dentro questa formula l'analisi di clos si vede che non prevede una probabilità pari a \(0\)

\subsection{Mi parli della frammentazione}
\label{sec:org7b544c9}
introdotta nell'ipv4
prevede il fatto che un pacchetto venga diviso in più pacchetti, che vengono poi enumerati per permettere la ricostruizione del pacchetto

se è connection oriented allora \ldots{}

\subsubsection{Mi dici quali campi del pacchetto sono utilizzati?}
\label{sec:orgdb14066}
qualcosa, quello più importante, che mi sono perso

l'identification che rappresenta l'ordine di sequenza del\ldots{} nel pacchetto e \ldots{}
associa il frammento a un puntatore per poterlo poi

c'è n'è poi un altro, son piuttosto banali, ma se uno non li dice
tra i due uno è più importante dell'altro

c'è un campo che dice se il frammento è l'ultimo della sequenza o no, per poterla finire
e un campo che dice se il frammento può essere ulteriormente frammentato
(ultimi due detti dal fantacci perchè non li sapeva)

c'è un qualcosa che aumenta il ritardo rispetto a quello che potrebbe essere allineandosi in maniera rigorsa quello che hai detto te
il ritardo che tutte le volte che si va a rifare la testata bisogna ricalcolare il checksum

\subsubsection{Nell'ipv6 come vengono risolti questi problemi?}
\label{sec:orgfecf157}
per avere un'elaborazione più veloce l'header dell'ipv6 è più piccolo
nell'ipv6 la frammentazione c'è o non c'è? c'è, le reti sono sempre quelle
però come viene implementata? Rispetto all'ipv4, in una manienra abbastanza semplice, viene stimata in una maniera end to end per determinare subito la dimensione massima di nodo e poi da lì si va a dritto
(sempre detto dal fantacci perchè lui non lo sapeva)

\subsection{Routing boradcast}
\label{sec:org0b8cda1}
ci sono varii metodi per farlo
una è avere tante connessioni unicast quanti i nodi della rete, ma bisognerebbe gestire gli stessi pacchetti molte volte
si ha \ldots{} tutti gli indirizzi dei nodi della rete, un nodo, letto il pacchetto, lo manda a tutti i nodi associati e \ldots{}

ok broadcast vuol dire tutti, tu stai descrivendo il multicast

\subsubsection{quali sono le tecniche di broadcast?}
\label{sec:orgbd167d3}
\begin{description}
\item[{flooding}] mandi a tutti i vicini salvo chi t'ha dato
\item[{\ldots{}}] qualcos'altro ma solo se \ldots{} costo minimo
\end{description}

ok qualche piccola incertezza, 28.

\section{N + 4}
\label{sec:org5155a49}
\subsection{parlami della tecnica aloha}
\label{sec:orga68c27c}
questo parla a voce troppo bassa perchè si capisca cosa sta dicendo
\subsubsection{con che criterio si sceglie un instante in cui trasmettere}
\label{sec:orge5dd143}
alla cazzo di cane, con probabilità uniforme per massimizzare la cazzo\footnote{per diminuire la probabilità che due collidano de novo}
(non lo sapeva)

\subsubsection{Ci sono due varianti, perchè lo slotted è meglio}
\label{sec:org9ea9301}
e cos'è che evita lo slotted che invece nell'aloha classico\ldots{}
diventa più piccolo, infatti qual'è l'evento che si esclude

in questo modo cos'è che eviti che succeda, eviti che \ldots{}, visto che saranno tutti allineati se vinci in quel momento vinci sempre
(non lo sapeva)

\subsection{Il conteggi all'infinito, che cos'è}
\label{sec:org444d64b}
il problema del conteggio all'infinito nasce quando si devono aggiornare le tabelle di routing con il meotodo di\ldots{}
(silezio abissale)
con il metodo distance vector (detta dal fantacci)

\subsection{Sicurezza a chiave pubblica}
\label{sec:org7fde494}
L'RSA, l'RSA è figo
\subsubsection{come funziona dal punto di vista di procedura?}
\label{sec:org96e998a}
proprio al livello di funzionamento a macrolivello

se due entità A e B vogliono parlare tra di loro A deve prendere la chiave pubblica di B
non l'algoritmo,
(non basta conoscere la chiave pubblica non si riesce a decifrare il messaggio perchè servono anche infromazioni dalla chiave privata che \ldots{})
(anche questa non lo sapeva)

\section{N + 5}
\label{sec:org7bb367d}
(comincia cancellando la lavagna, promettente)
(parla molto in fretta, elevato package loss)
\subsection{Problema del terminale esposto, e come viene risolto}
\label{sec:org749eb6e}
intanto siamo nell'ambito delle reti wireless
il problema del terminale esposto è innanzitutto da ricondurre alla soluzoine del problema del terminale nascosto

(diagramma eulero venn con i raggi di a, b, c, d)
a parla con B e fa l'handshake
il messaggio viene ricevuto da tutti i nodi nel raggio di A
questo imposta tutti i valori di NAV nell raggio di A, quindi anche quello di B

(inizia a non tornarmi la nomenclatura, sta confondendo un minimo B con C mi sa)
(mette che A può dialogare con C)
(ok corregge la nomenclatura)

quindi D è il terminale esposto

\subsubsection{Ok, e come si risolve?}
\label{sec:org274b52b}
una strategia più che per risolverlo è che per mitigarlo è il \ldots{} off
una soluzoine alla radice cosa potrebbe essere? si è detto a lezione e sul libro
non c'è una soluzione alla radice

\subsection{Tecnica diffuzion boh}
\label{sec:orgb6bee01}
usata nelle reti di sensori
associa a ogni \ldots{} una coppia attributo valore
trasmessa con tecnica flooding
cosa viene inviato, spesso una lista di attributo valore

inizia con manifestazione di interesse da parte del mittente
poi \ldots{} gradient per \ldots{}
raccolta dato da parte del sensore lungo il percorso scleto dal \ldots{} gradiente
poi reinforcement, si usa per uno scopo ben preciso, non il rate della trasmissione, c'è una ragione molto pratica per cui il nodo sink fa reinforcement
perchè in questo modo ne sceglie una tra tutte così gli altri non\ldots{}

\subsection{Sliding window}
\label{sec:org774e6a6}
tutto è architettato sulla base del numero di sequenza, anche per il numero di riscontro
vado a definire una \(wl\) o \emph{window lenght}, quanti pacchetti possono essere mandati di fila
entro la finestra deve essere stato mandato il riscontro del primo pacchetto della finestra
quando il riscontro del primo pacchetto avviene per tempo allora non siamo in uno stato di congestione
quando non è per tempo siamo in uno stato di congestione, la finestra viene chiusa e sarà riaperta quando arriva il pacchetto

ogni volta che \ldots{} si va ad aumentare di uno la lughezza della finestra, questo per testare la rete fino a quando non arriva una \ldots{} di congestione
poi si prende un valore che è la metà esatta del valore che ha creato la congestione e poi si va nella fase di congestion avoidance
(il fantacci c'ha na faccia mo' proprio)

\ldots{}
si va a scegliere una lughezza della finestra di valore pari alla metà del\ldots{} questo va bene, poi
poi si va ad inviare i pacchetti, ma con che procedura, si riinizializza la finestra a 1, e quando si arriva al valore target determinato come metà del valore di congestione è lì che parte il meccanismo di congestion avoidance

27, è andato bene

\section{N + 5}
\label{sec:orgf73cfa6}
(mancato)

\section{N + 6}
\label{sec:orgd7f1321}
\subsection{Strutture di commutazione a divisione di tempo}
\label{sec:orgcc39fe2}
come sono fatte
qual'è il costo
\ldots{}

le strutture a commutazione a divisione di tempo, chiamate strutture \(T\), sono un tipo di struttura di commutazine, poi ci sono quelle \(S\)
sono implementate con delle memorie
sono utilizzate soltanto nella telefonia numerica e permettono la permutazione di canale
in un tempo di trama, fissato per convenzione a \(125 \mu s\)
questo poi va diviso per \(2n\), con \(n\) numero di trame, (visto che accesso sia in lettura che scrittura per ognuno)
il tempo di \(125 \mu s\) è scelto perchè bla bla bla campionamento shannon

\begin{itemize}
\item scrittura casuale lettura sequenziale
\item scruttura sequenziale lettura casuale
\end{itemize}

(si usa \emph{casuale} perchè potrebbe essere qualsiasi ordine)

\subsection{Congestion Avoidance}
\label{sec:orgcc04856}
è una tecnica reattiva della sliding window
\ldots{} quando si è a una dimensione matura della rete, a differenza dello slow start, invece di aumentare esponenzialmente la dimensione della finsestra (\texttt{std::vecotr} time)
si va in maniera più prudente per \ldots{}

\subsubsection{che alternative si hanno?}
\label{sec:orge03f18a}
possiamo reiniziare con un altro slow start per \ldots{}

\subsection{Conteggio all'infinito}
\label{sec:org08fe3d6}
il conteggio all'infinito è un problema che si crea nell'algoritmo di distance vector quando un \ldots{} non esiste più tra due host
l'host \ldots{}
il collegamento tra \(B\) e \(C\) non esiste più
ma \(A\), collegato a \(B\), e che usa \(B\) per arrivare a \(C\), manda a \(B\) il fatto che "hey ci arrivo con \(AB + BC\)", \(B\), che pensa quel valore non lo riguardi, se lo salva.
poi quando rimanda il valore ad \(A\) allora \(A\) incrementa il suo costo per \(C\), lo rimana a \(B\), che poi lo manda ad \(A\) che incrementa il suo costo per \(C\), che al mercato mio padre comprò

\subsubsection{Soluzoini}
\label{sec:org4ce9194}
\begin{itemize}
\item infinito finito
\item split horizon
\item \ldots{}
\end{itemize}
\ldots{}
quello non basta, ci vuole qualcosa di più a monte.
si evita di mandare dei cammini che si sa saranno attivati partendo dal nodo a cui si vuole mandare\ldots{}
quidni qui \(A\) non manderà a \(B\) niente che riguardi \(C\), questo è un metodo, e l'altra

questa tecnica apparentemente risolve tutto, ma c'è un inghippo per come in questa rete si gestisce l'aggiornamento

può darsi che una rotta sia ancora considerata valida e \ldots{}

se non c'ho nulla entro un tempo si assume che il nodo non faccia più parte della rete
l'ultima variante cosa fa per risolvere questo problema? Il nodo manda \ldots{} per rinfrescare il collegamento e mantenerlo attivo

30 vai

\section{N + 7}
\label{sec:org35104f8}
il veterano
\subsection{Parlami delle tecniche ADSL}
\label{sec:org2cd93e0}
le tecniche di adsl sono tecniche utilizzate per il trasferimento di dati di più tipologie
\begin{description}
\item[{ci sono quelle simmetriche}] tengono la stessa banda per upload e download
\item[{ci sono quelle asimmetriche}] con una preferenza per il download, o per l'upload
\begin{itemize}
\item la preferenza per il \textbf{download} è più per \textbf{roba domestica}
\item per i \textbf{server e roba} a cui si richiede tante cose c'è più in \textbf{upload}
\end{itemize}
\end{description}

come si fa \ldots{} con il coso di rame dial up
il coso di rame \ldots{} una direzione alla volta
\ldots{} telefonia numerica \ldots{}

questa tecnica è \ldots{}
però in ricezoine come si fa a dividere parte utente da\ldots{}
si fa con un filtro passa basso e con un filtro passa alto

l'adsl ha la caratteristica, un po' più tecnica, di dare velocità più alta a \ldots{} perchè adatta la modulazione \ldots{} canale

\subsection{Nat, a che serve, come funziona}
\label{sec:orgef80a22}
ho delle lan con un indirizzo
un nat serve a specificare a quele dispositivo

serve a installare con una connessione da questo a questo senza passare da intenret
ed esporre la lan a internet con una tecnica (\ldots{})

serve ad aggregare più utenti con un solo indirizzo ip
normalmente non ci sono conflitti, ma metti ho con lo stesso numero di porta indicato da due\ldots{}
il nat ha una tabella interna e boh

il router nat va a scegliersi una tabella di routing interna associandola a boh.
quando \ldots{} ricostruisce il datagramma corretto e lo manda a quello giusto

\subsection{Tecniche polling}
\label{sec:org786b671}
non so cosa cazzo ha detto ma il fantacci si è tolto gli occhiali in modo drammatico

ha preso 22

\section{N + 8}
\label{sec:org2a92b68}
\subsection{Strutture di commutazione S-S}
\label{sec:org674ee2f}
sono delle strutture omogenee
consentono il cambio di linea
\begin{itemize}
\item si partizionano le linee di ingresso in più blocchi con lo stesso nuemero di ingressi
\item si partizionano le lineee di uscita in più blocchi con lo stesso numero di uscite

\item la seconda del primo (ingresso) viene collegata alla prima del secondo (uscita) e così via
\end{itemize}

\subsubsection{Bloccaggine}
\label{sec:orgde06ba4}
sono bloccanti
per renderle non bloccanti ogni nodo di quelli in ingresso dovrebbe avere tante uscite quante quelle della rete

descrivi quando nasce il blocco e poi dimmi come si fa ad evitare
(qualche incertezza nell'esposizione di questo punto)

\subsection{Router generalizzato}
\label{sec:org7f63742}
\ldots{}

si suddividono in due famiglie
\begin{description}
\item[{senza tabella}] flooding et al
\item[{con tabella}] tecniche
\begin{description}
\item[{statiche}] \ldots{}
\item[{dinamiche}] \ldots{}
\end{description}
\end{description}

e invece il fantacci stava parlando delle sdn, software d(?) network
il routing gnenralizzato consiste nell'instradamento e \ldots{} di pacchetti
aspe' questo non è \ldots{} questo è come è struttrata la rete

il routing generalizzato è a livello più basso, perchè si chiama generalizzato
perchè si va a fare il routing non solo con la coppia sorgente destinazione, ma anche valori in altri campi, tipo al livello tcp
per fare ad esempio routing in modo diverso per divesi tipi di traffico 
si va ad aumentare i campi da leggere e da processare

\subsection{Tecnica (dip)?}
\label{sec:orga851653}
dinamica
"imparentata" con il distance vecotr
usa come valore di irrangiugibilità il numero 15

come metrica per il costo dei singoli collegamenti usa il numero di \ldots{}

25
(ci pensa)

\section{N + 9}
\label{sec:orgba6ff6d}
\subsection{Reti FDDI}
\label{sec:org09ec889}
hanno connessioni su bus
sono entrambi direzionali

quella principale è quella esterna
quella interana è di backup e/o per aumentare la banda

c'è il token, ci sono contatori

partiamo dall'inizio, per funzionare qui si va a definire un parametro di riferimento,
il \texttt{Token Target Rotaton Time}

il valore effettivo sarà maggiore o ugugale del tempo target, torvato questo valore come si usa per gestire l'accesso
tieni il token e mandi il token e \ldots{}

per arrivare a \ldots{} il nodo deve \ldots{} alcune operazioni perliminari
cosare il tempo effettivo per \ldots{} cosare

si va a calcolare il \texttt{Token Holding Time}, il tempo per cui è concesso tenere il token
se \ldots{} è maggiore del tempo che ci vuole a \ldots{} può mandare anche asincrona
altrimenti il tempo di riferimento sarà solo \ldots{} in maniera sincrona

\subsection{Tecnica token bucket}
\label{sec:org0f0b3ee}
utilizzata sempre per l'accesso oridnato al canale e \ldots{}

cos'è innanzitutto, è una tecnica per la prevenzione delle congestioni
\begin{itemize}
\item si ha un buffer, il token bucket
\item e un altro buffer, i pacchetti da mandare
\end{itemize}

così si tengono i permessi del nodo per mandare la roba che manda
che parametro controlla, il rate.
il rate è una variabile aleatoria, questa tecnica quale aspetto del rate tiene costante (va a controllare il rate medio)

\subsection{Maschera di rete}
\label{sec:org0ec2a64}
una maschera di rete serve a indicare quali bit sono per l'indirizzo della rete e quali bit sono per l'indirizzo dell'host
nella maschera viene dato un ulteriore numero e \ldots{}

il router come fa per scoprirlo? (mi sono perso questa parte)

24

\section{N + 10}
\label{sec:org13ab045}
(package lost)

\section{N + 11}
\label{sec:org94ceecf}
\subsection{Strutture di commutazoine S-T}
\label{sec:org8300e8c}
sono strutture di commutazione non omogenee
in genrale queste strutture permetono sia un cambio di canale all'interno della stessa linea che un cambio di linea

per risolvere questo problema di blocco si usano strutture a tre stadii, come la T-S-T o la S-T-S

\subsubsection{Problema di blocco}
\label{sec:orgebac320}
il blocco succede quando ho due linee da due canali di ingresso che vogliono andare su due canali diversi della stessa linea di uscita

\subsection{Accesso ordinato}
\label{sec:orgcd90f1f}
a livello MAC
\begin{description}
\item[{Roll Call}] il master ha una lista, questa può essere utilizzata per tipologie a stella
\item[{Hub Polling}] per tipologie a bus, il master, a un estremo del bus di solito. manda un \ldots{} all'ultimo nel bus, e questo lo porta indietro
\end{description}

\subsubsection{Tipi di accesso}
\label{sec:org1d50b00}
\begin{description}
\item[{gated}] si ha un tempo massimo, se hai pacchetti da mandare dopo quel tempo aspetti
\item[{esaustivo}] il nodo può trasmettere quanto vuole finchè non ha finito i pacchetti
\end{description}

e questo tempo gated come viene definito?
durante una fase di setup della rete? NO
è un tempo casuale che dipede da varii parametri relativi al nodo e alla rete

\subsection{Tecniche Clustering}
\label{sec:org0b4d519}
utilizzato nelle reti di sensori per rendere le reti di sensori più efficienti
\begin{description}
\item[{leach}] \texttt{low energy adaptive clustering hierarchy}
si ha un capo cluster\ldots{}
in generale qui tutti devo essere connessi al sink in quanto devono poter essere tutti capo cluster
il problema con queste \ldots{}
non mirata all'energia
\item[{HEED}] Hyprid Energy Efficient Distributed Clustering
\ldots{}

ti do un voto che in genere non do mai, 29
\end{description}

\section{N + 12}
\label{sec:org207f702}
(di elettronica e automazione)
\subsection{Formula di Clos e ottimizzazoine del costo}
\label{sec:org8306da7}
si calcola il numero di blocchi \ldots{} del secondo stadio
\ldots{}
si analizza il caso peggiore
\begin{itemize}
\item i due blocchi sono disgiunti
\item si fanno altre ipotesi e boh
\end{itemize}

il calcolo del costo non me lo ricordo

\subsection{CSMA CA}
\label{sec:orgf1dc247}
tecnica di accesso casuale che prevede
acrnonimo sta per
\begin{itemize}
\item carrier
\item sensing
\item \ldots{}
\end{itemize}

si divide in più tipi
\begin{description}
\item[{persistent}] ascolta di continuo e manda quando libero
\item[{persistent p}] ascolta e quando libero manda con probailità costante \(p\)
\item[{non persistent}] ascolto \ldots{}
\end{description}

(che poi aveva chiesto il csma ca perchè hai detto del csma in generale)

\subsection{Problema del terminale nascosto}
\label{sec:org5ce1429}
abbiamo i terminali \(A\), \(B\), e \(C\).
si ha una collisione quando \(A\) manda a \(B\) e \(C\) non riesce a vederlo

si risolve implementando una procedura di handshake con \(RTS\) e \(CTS\)

i pacchetti \(RTS\) e \(CTS\) hanno un nav
il nav è un \ldots{} che inibisce \ldots{} per permettere la durata massima della trasmissione

in nav serve a dare una durata all'inbizione dei terminali per evitare collisioni

\subsection{Congestion avoidance fast recovery}
\label{sec:orgfcd52bb}
si divide in \ldots{}
c'è il caso della congestione lieve e il caso della collisione critica

\subsubsection{lieve}
\label{sec:org55997ae}
la congestione lieve si ha quando un pacchetto del quale non si ha riscontro e successivamente a questo mancato riscontro arrivano altri pacchetti
arrivano riscontri di pacchetti ma non quelli dei pacchetti che si stanno aspettando

fino a quando si può attendere per questo?
\begin{itemize}
\item fin quando non scade il tempo massimo per cui \ldots{}
\item o se invece questa cosa non è avvenuta però? Arrivano più di 3 pacchetti fuori sequenza, in tal caso la condizione è grave
\end{itemize}

\subsubsection{critica}
\label{sec:org5356195}
(mi sono perso questa parte, credo si torni allo slow start)

24

\section{N + 13}
\label{sec:orgd0494bd}
\subsection{nel bridge}
\label{sec:org47921a7}
\subsubsection{Come si risolvono i cicli}
\label{sec:org354bc87}
in quel caso invece di \ldots{} si fa un albero
solitamente si crea utilizzando il minimun spanning tree fatto con dijkstra

\subsubsection{Fino a che livello opera?}
\label{sec:orgea6b948}

\subsection{Routing senza tabella}
\label{sec:org21f8f23}
\begin{description}
\item[{casuali}] si basano sulla scelta, il metodo di frowarding, su una distribuzoine statistica, mandano fuori sulla porta con maggior\ldots{}
\item[{flooding}] lo mandi a tutti salvo quello che te l'ha mandato
\end{description}

il flooing è motlo costoso, ci sono accorgimenti che si possono fare al riguardo, come mettere un parametro \texttt{TTL}, o \texttt{Time To Live} nell'header

\subsection{??}
\label{sec:org3ce8579}
altrimenti si fa che mandi un messaggio in flooding che tiene conto del numero di hop fatti e degli hop che ha fatto, il primo messsaggio flooding arrivato sarà quello col cammino minimo, qiundi si rimanda indietro il pacchetto col cammino che ha fatto e \ldots{}

\subsection{il TCP e come apre connessioni}
\label{sec:org96a2115}
\ldots{}
il tcp manda in maniera ordinata, il destinatario ordina con un buffer
con riscontro

per l'apertura del percorso si fa con una techica di handshaking
nel tcp i pacchetto contengono un numero di sequenza e un numero di riscontro

il numero di sequenza non parte sempre da 0 ma si sceglie un intero 32 bit alla cazzo

\subsubsection{Handshake}
\label{sec:org3c705ff}
nell'handshake ci sono quei tre messaggi
si hanno falg

\begin{enumerate}
\item si inizia col messaggio col flag \texttt{Syn} e un numero di sequenza
\item il server risponde con un messaggio \texttt{Ack} con un numero di sequenza e il riscontro del \texttt{Syn}
\item non ha il tempo di finire la frase che prende 30
\end{enumerate}

\section{N + 14}
\label{sec:org964092b}
\subsection{Indirizzamento IPv4 con le classi}
\label{sec:org001e997}
l'ipv4 fa parte del tcp ip che fa parte dell'IEEE 802
l'ipv4 \ldots{} 160 bit e ha un \ldots{} limitato che altrimenti viene frammentato
nell'header ci sono i flag sulla frammentazione

\subsubsection{L'indirizzamento per classi?}
\label{sec:orgc4bc4c4}
innanzitutto le classi cosa fanno?
partizionano il campo di 4 byte in sottoparti con dimensioni vincolate
hai la parte a net id e la parte a host id

il problema però di questo è che se uno prende un blocco di indirizzi ti prendi tutti gli host id sottostanti, e se non li usi tutti è uno spreco
allora è stato trovato il modo della subnet mask eccetera eccetera
(tutto questo detto dal fantacci per motivi che il lettore capirà bene)

\subsection{il protocollo UDP}
\label{sec:org00ef35f}
l'udp fa parte del layer di trasporto, come il tcp
ma è un protocollo, a differenza del tcp, di tipo connection less (i datagrammi fanno il cazzo che vogliono)
(è pure inaffidabile)
(molta incertezza sull'esposizione)

come si fa a renderlo connection oriented? una furbata
si fa con un solo pacchetto, e allora pefforza la connessinoe l'ordine è lo stesso dell'invio

\subsection{CLOS T-S-T}
\label{sec:org1427996}
la T-S-T permette di cambiare la linea sia prima che dopo la S
come si riadattano le ipotesi del caso S-S-S per il caso T-S-T (di clos o di costo boh)
(altre incertezze sull'esposizione)

voto 23

\section{N + 15}
\label{sec:org2b3e465}
\subsection{Compatibilità tra IPv4 e IPv6}
\label{sec:orgf7a1f30}
un'area con un tipo di rete attraversa un'area con un tipo di rete diverso
c'è un riadattemento del tipo di protocollo

ad esempio per intepretare un ipv6 e mandarlo su una rete solo ipv4
si è fatta l'analogia \emph{attraente} del traghetto che porta le macchine sul fiume

l'altro più semplice da dire e più complicato da fare è quello dual stack

\subsection{Reti SDN}
\label{sec:orgbeaca28}
so\ldots{} defined network

si separa il piano controllo dal piano hardware e si utilizzano metodologie di "virtualizzazoine"
abemus virtual machine
\subsubsection{Perchè il problema del conteggio all'infinito non è presente?}
\label{sec:orgd1d3d2e}
perchè qui il problema col distance vector non c'è
non c'è l'aggiornamento continuo tra nodo a nodo che porta a quel problema
cosa c'è qui in sostutizione a questo aggiornamento da nodo a nodo? C'è il \emph{famoso} nodo centrale

\subsection{Leaky bucket}
\label{sec:orgeea4c9b}
se hai trasmetti, se non hai non trasmetti
questa tecnica controlla esattamente il rate massimo
non posso mandare più di un pacchetto ogni \ldots{} secondi
(il bitrate medio è del token bucket)

23

\section{N + 16}
\label{sec:orgf3083dd}
(altre stampelle, quindi si eviterà la lavagna)
(ingengeria informatica)
\subsection{Strutture T-S}
\label{sec:orgfa802dd}
sono strutture per avere più libertà rispetto alle strutture T o S singole
le S sono utilizzate per telefonia analigoca e digitale
le T sono utilizzate per telefonia digitale e basta
\subsubsection{quando sono bloccanti}
\label{sec:org81d15d8}
il concetto di non bloccante vuol dire che in ogni momento puoi prendere un qualsiasi ingresso e metterlo su qualsiasi uscita

quando su una linea di ingresso che ha una struttura T ci sono due richeste che vogliono andare sulla stessa posizione di trama su due linee di uscita diverse

\subsection{NAT}
\label{sec:orgdf8dc92}
introdotto dopo l'ipv4 visto che non ci sono abbastanza indirizzi ipv4, vuol dire
\begin{itemize}
\item network
\item address
\item translation
\end{itemize}

ci sono
\begin{itemize}
\item nat statica
\item nat dinamica
\item pat
\end{itemize}

\subsubsection{Statica}
\label{sec:orge19e8af}
ha una tabella in out di indirizzi 

\subsubsection{Dinamica}
\label{sec:org5c533c9}
non è la pat, ma sono simili
nella nat c'è solo un indirizzo pubblico, se chiedono entrambi lo stesso servizio non si sa a chi assegnarlo
il problema sorge più che altro quando usano poi entrambi lo stesso numero di porta che il traffico diventa 

\subsubsection{Pat, port address translation}
\label{sec:org6bac880}
si ha un solo indirizzo pubblico
qui all'indirizzo interno si associa una porta privata
si fa un po' di routing a livello trasporto, balsfemo proprio

\subsection{Tecnica RPF}
\label{sec:orgd6a0228}
\begin{description}
\item[{R}] 

\item[{P}] ath
\item[{F}] orwarding
\end{description}

sensori
di dati basata su \ldots{} boh   

un nodo prima di fare una certa cosa verifica un'altra
che il cammino da sorgente e \ldots{} sia il minimo possibile
se non è arrivato dal cammino minimo il pacchetto viene scartato, così facendo si alleggerisce la rete

25

\section{N + 17}
\label{sec:org12c8d87}
ultimo della mattinata, ingnegneria informatica
\subsection{Ottimizzazoine del costo di una struttura S-S-S di CLOS}
\label{sec:org9c1555c}
il risultato di clos si attua operando nel caso operativo peggiore
si trova il \(k\) minore che ha comunque una condizione di non blocco

ci sono delle ipotesi che definiscono il caso peggiore
\begin{itemize}
\item un solo ingresso libero
\item una sola uscita libera
\item la roba del secondo livello che prende il primo e che prende il secondo sono disgiunti
\end{itemize}

ok ecco le ipotesi
\subsubsection{Ottimizzalo}
\label{sec:orgbe800c5}
\begin{itemize}
\item \(N\) è il numero di linee totali in ingresso e in uscita
\item i blocchi di ingresso sono \(n\), i blocchi di uscita sono \(n\)
\item ci sono \(k\) blocchi nel secondo stadio, che saranno \(frac{N}{n} \times \frac{N}{n}\)
\end{itemize}

roba, poi si ipotizza che \(N >> 1\), quindi \ldots{}, e ottenuta la formula del costo si fa la derivata e si mette pari a 0 per trovare il minimo.

\subsubsection{Vincoli realizzativi}
\label{sec:org7f1133b}
questo \(n\) dovrebbe essere un numero intero, quindi si va a vederlo in eccesso o difetto per vedere quale dei due ha costo minimo

\subsection{Indirizzo class dell'ipv4}
\label{sec:org11d2f35}
ci sono 5 classi, \(A\), \(B\), \(C\), \(D\), ed \(E\).
la classe \(E\) è riservata per indirizzi futuri
le altre classi sono per tipi di traffico diversi
\begin{description}
\item[{\(A\)}] je
\item[{\(B\)}] ne
\item[{\(C\)}] sais
\item[{\(D\)}] pas
\end{description}

\subsection{CSMA}
\label{sec:orgae03096}
\ldots{} , si differenziano da aloha visto che stanno in ascolto del canale
in caso si verfichi una collisione la di gestione si fa con una fase di backoff in cui aspetti un pochino alla cazzo
a volte due nodi mandano insieme perchè non si sono sentiti a vicenda
si vuole fare in modo che il tempo di vulnerabilità sia molto minore al tempo di inviare il messaggio

voto 30
\end{document}