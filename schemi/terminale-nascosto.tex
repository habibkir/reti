% Created 2023-07-23 Sun 19:52
% Intended LaTeX compiler: pdflatex
\documentclass[11pt]{article}
\usepackage[utf8]{inputenc}
\usepackage[T1]{fontenc}
\usepackage{graphicx}
\usepackage{longtable}
\usepackage{wrapfig}
\usepackage{rotating}
\usepackage[normalem]{ulem}
\usepackage{amsmath}
\usepackage{amssymb}
\usepackage{capt-of}
\usepackage{hyperref}
\author{Biggie Dickus}
\date{\today}
\title{}
\hypersetup{
 pdfauthor={Biggie Dickus},
 pdftitle={},
 pdfkeywords={},
 pdfsubject={},
 pdfcreator={Emacs 28.2 (Org mode 9.5.5)}, 
 pdflang={English}}
\begin{document}

\tableofcontents

wireless

\section{Terminale Nascosto, da whatsapp}
\label{sec:orgb520203}
\subsection{Problema}
\label{sec:org7d8ac78}
abemus A, B, e C
A comunica con B, C vuole comunicare con B
C è fuori dalla portata di A, e non sa che A sta parland con con B
C manda il suo pacchetto a B, che riceve sia da A a che da C, collisione su B, che non capisce un cazzo e perde i pacchi

\subsection{Soluzione}
\label{sec:orgc441f11}
imposta in tutti i nodi una three way handshake un pacchetto RTS e uno CTL

quindi adesso quando A manda a B prima deve
\begin{itemize}
\item mandargli un RTS
\item e se B è libero, B riceve l'RTS, e manda a tutti i terminali nel suo range un CTL
\item A, ricevendo il CTS, sa che B è libero, e manda
\item C, che ha ricevuto il CTS senza aver mandato un RTS, sa che B è occupato, e aspetterà il suo turno
\end{itemize}


\section{Terminale Esposto, da whatsapp}
\label{sec:org7ae395b}
la soluzione al problema del terminale nascosto porta alla creazione del cosiddetto \emph{problema del terminale esposto}, vale a dire

\subsection{Problema}
\label{sec:org3dacb98}
C riceve il CTS da B, e se dovesse ricevere un RTS da D, altro terminale a cazzo allora questi due, (RTS di D e CTS di B) si annullano e C non manda un cazzo
visto che D sta apsettando un CTS per mandare, e su C collisione a quanto pare, allora D non riceve il CTS da C e non manda un cazzo pure lui

\subsection{Soluzione}
\label{sec:orgd4e0092}
questo si risolve impostando un tempo di back-off che poi viene messo nel NAV di ogni nodo


\section{Terminale Nascosto, Fantacci}
\label{sec:org0e6927c}
procedura di handshake (riscritta in C++ perchè i diagrammi di flusso mi fanno gottare)
\begin{verbatim}
void Terminale::aspetta_alla_cazzo() {
    this->wait_time(IFS);
    int r = rand();
    this->wait_time(r);
}

void Terminale::manda_e_aspetta(Canale* c) {
    this->send_frames_on_channel(c);
    this->wait_time(SIFS);
}

Paccketto Terminale::ricevi_ack(Canale* c) {
    this->manda_e_aspetta(c);
    return this->wait_for_package(c);
}

void Terminale::accesso_canale(Canale* c) {
    int volte_che_non_mhanno_cagato = 0; // k
    this->wait_for_free_channel(c);
    this->wait_time(DIFS);
    if(c->is_free()) {
        this->aspetta_alla_cazzo();
    }
    this->manda_e_aspetta(c);
    if(this->ricevi_ack(c) == Pacchetto::due_di_picche) {
        volte_che_non_mhanno_cagato ++ ;
    }
    else {
        this->boh_apro_connessione(c);
        return;
    }

    /* e ora comincia il ciclo di non farsi cagare */
    while(volte_che_non_mhanno_cagato < 15) {
        this->aspetta_alla_cazzo();
        this->manda_e_aspetta();
        if(t->ricevi_ack() == Paccketto::due_di_picche) {
            volte_che_non_mhanno_cagato ++;
        }
        else {
            this->boh_apro_connessione(c);
            return;
        }
    }
    std::cout<<"mai na gioia, mi arrendo"<<std::endl;
    this->piangi_in_un_angolino();
    return;
}
\end{verbatim}

\subsection{handshake}
\label{sec:orgf998868}
la fase di handshake ha tre fasi
\begin{itemize}
\item conclusa la fase di accesso, se il canale è ancora libero, \(A\) manda un messaggio \(RTS\) richiesta di collegamento a \(B\)
\item se \(B\) riceve l'\(RTS\) correttamente allora aspetta \(SIFS\) e manda un messaggio di conferma (\(CTS\)) ai vicini
\item \(A\) riceve il \(CTS\) da \(B\) e, atteso un tempo \(SIFS\), sa che adesso può mandare roba a \(B\).
\end{itemize}

conclusa la fase di handshake la gestione delle successive fasi di accesso avvengono secondo normalissimo manda e \(ACK\)    

\section{Terminale Esposto, Fantacci}
\label{sec:org710b3cd}
\subsection{Contesto}
\label{sec:org428ee5f}
nel problema del terminale nascosto

l'invio del messaggio \(RTS\) viene visto da tutti i terminali nel raggio di \(A\), che vedono un gigantesto "\(A\) vorrebbe parlare con \(B\)" e pensano "ok, eviterò di farmi i cazzi di \(A\) o \(B\) per ora"

il messaggio "ciao sono \(A\) vorrei parlare con \(B\)" non viene visto da \(C\) visto che \(C\) non è nel raggio di \(A\).

quello che \(C\) vede sarà il \(CTS\) di "ciao sono \(B\) e mi metto a parlare con \(A\)", con cui sa che "ok, non mi farò i cazzo di \(B\)"

\subsection{Problema}
\label{sec:orgaa76c17}
il ricevimento da parte di \(C\) del messaggio \(CTS\) gli impedisce di inviare o ricevere per un certo tempo (quello nel campo duration id).
ci si accorge subito che utilizzare un handshake abbia effetti sull'efficienza della rete.
si riducono le collisioni ma ci vuole più tempo a cominciare a mandare roba, per limitare l'efficienza o meno della rete occorre che i frame da mandare siano abbastanza lunghi da rendere trascurabile i tempi di setup (e/o teardown)

\subsubsection{Inibizione dei pacchetti post \(CTS\)}
\label{sec:org4c11df8}
ok, abbiamo un terminale che ha ricevuto un \(CTS\) uscito da una handshake che non lo riguardava e ora sta zitto
per evitare che i terminali provino ad attivare fasi di attivazione canale quando questo è già attivo è stato introdotto in ogni terminale un ulteriore contatore a decremento chiamato \texttt{Network Allocation Vector}\footnote{il contatore si chiama vettore, ma vaffanculo}

si è visto che il header \(MAC\) contiene un campo che indica la durata della trasmissione un duration id, appena un terminale vede invia un \(RTS\) o \(CTS\) , tutti i nodi nella sua area di copertura leggono questo campo e impostano il loro \(NAV\), come "timer de li cazzi tua", al duration id del header mac.

poi solito di prima, \(C\) riceve il "fatti li cazzi tua", \(D\) non è in range, prova a contattare \(C\), e \(C\) sta aspettando il \(CTS\) quindi non si accorge dell'\(RTS\) di \(D\), tutti stanno male, amen.

\subsubsection{Monopolio}
\label{sec:org1eb7961}
Un altro problema che può sorgere è che certi canali poi abbiano un accesso esclusivo al canale per troppo tempo a discapito di altri nodi della rete

lo standard prevede quindi la possiblità di assegnare a un nodo anche un tempo massimo di trasmissione, indicato come Transmission Opportunity
\end{document}