% Created 2023-07-23 Sun 19:54
% Intended LaTeX compiler: pdflatex
\documentclass[11pt]{article}
\usepackage[utf8]{inputenc}
\usepackage[T1]{fontenc}
\usepackage{graphicx}
\usepackage{longtable}
\usepackage{wrapfig}
\usepackage{rotating}
\usepackage[normalem]{ulem}
\usepackage{amsmath}
\usepackage{amssymb}
\usepackage{capt-of}
\usepackage{hyperref}
\author{Biggie Dickus}
\date{\today}
\title{}
\hypersetup{
 pdfauthor={Biggie Dickus},
 pdftitle={},
 pdfkeywords={},
 pdfsubject={},
 pdfcreator={Emacs 28.2 (Org mode 9.5.5)}, 
 pdflang={English}}
\begin{document}

\tableofcontents

\section{accesso multiplo}
\label{sec:org8b907e0}
condivisione del mezzo fisico
puoi farlo con un ordine o alla cazzo
\subsection{Ordine}
\label{sec:org81057b0}
\begin{description}
\item[{fdm tdm}] multiplexing per tempo o frequenza, infefficienti nelle lan perchè portano ad accesso discontinuo
\item[{cdma}] code diviion multiple access
ogni utente trasmette con tutta la banda del canale
il ricevitore estrae solo la componente di interesse
\item[{ofdma}] orhtogona frequency division multiple access
l'acronimo si desccrive da solo
\item[{polling}] \begin{description}
\item[{roll call}] centralizzato, master gestisce autorizzazione e rilascio della risorsa canale
\item[{hub polling}] cooperazione tra nodi
\end{description}
\end{description}

poi i tipi di accesso possono essere
\begin{description}
\item[{gated}] hai il canale con una time slice massima
\item[{esaustivo}] hai il canale finchè hai roba da mandare
\end{description}

altri tipi di accesso sono
\begin{description}
\item[{token passing}] implementazione distribuita, ci si passa la chiave (token) in giro
\end{description}

\subsection{Alla cazzo}
\label{sec:org1860b7d}
\begin{description}
\item[{aloha}] nessun controllo sullo stato del canale
\begin{description}
\item[{puro}] nessun cordinamento
mandi il pacchetto e aspetti un riscontro
se non arriva il riscontro aspetti un tempo alla cazzo\footnote{con distribuzione uniforme in quanto così non ci sono tempi privilegiati che\ldots{}} e poi rimandi il pacchetto, sperando che stavolta funzioni
\item[{slotted}] il tempo è divsiso in slot lunghi quanto il tempo di trasmissione
i nodi possono mandare solo all'inizio di questi slot
così quelle collisioni che avvengono sono più contenute
collisioni gestite come nell'aloha puro, ma comunque slotted
\end{description}
\item[{CSMA}] carrier sense multiple access volto a ridurre le collisioni, si controlla che il canale non sia occupato prima di manda roba
si può intepretare questo come collisione virtuale e riprovare dopo un tempo alla cazzo.

Alcuni tipi di CSMA
\begin{description}
\item[{1-persistent}] un solo nodo in ascolto costante, funziona con pochi nodi
\item[{non persistent}] 

\item[{p-persistent}] si accede al canale su base statistica, ogni tanto alla cazzo
\item[{cd}] (collision detection) si sta in ascolto sul canale anche durante la trasmissione per controllare che non arrivino flussi mentre sto fluendo
\end{description}
\end{description}

\section{Ethernet}
\label{sec:orgd2f02e6}
può avere topologia a bus o stella
usare cavo coassiale, doppino telefonico o fibra
sviluppata perchè permette rate elevati ed è più flessibile di alternative quali FDDI

\subsection{Livello MAC}
\label{sec:orga338127}
gestisce l'accesso al canale con csma cd modalità 1-persistent
tutte le tecnologie ethernet hanno a livello di rete un servizio connectionless, quindi niente handshake
se il frame ricevuto non è pertitnente (i byte di check crc non concordano) il frame viene scartato

\subsubsection{Frame ethernet}
\label{sec:org1f87f70}
\begin{description}
\item[{preambolo}] 7 byte
\item[{delimitatore}] un byte
\item[{indirizzo destinazione}] 6 byte
\item[{indirizzo mittente}] 6 byte
\item[{lunghezza/tipo}] 2 byte
\item[{dati}] 

\item[{crc}] 4 byte
\end{description}

formato sempre quello
lughezze del frame minima 64 byte e massima 1518 byte, quella massima per gestire i buffer e non monopolizzare il canale

\subsection{Livello fisico}
\label{sec:org93a9753}
ci sono varii ethernet e ognuno fa li cazzi sua con il livello fisico
\begin{description}
\item[{standard}] 10 mbps, implmentazioni
\begin{description}
\item[{10base5}] coassiale grosso, bus
\item[{10base2}] coassiale sottile, bus (lughezza massima minore)
\item[{10base-T}] doppino telefonico, stella
\item[{10base-F}] fibra ottica, stella
\end{description}
\item[{veloce}] 100 mbps, retrocompatibile
\item[{gigabilt}] 1 gigabit, retrocomatibile
\item[{10 gigabilt}] 10 gigabit, retrocomatibile
\end{description}

\subsection{dispositivi di connessione}
\label{sec:orge661a3b}
per far comunicare lan distinte o connettere la lan alla rete globale si usano
\begin{description}
\item[{a livello applicazione}] gateway
\item[{a livello trasporto}] gateway
\item[{a livello rete}] gateway e router
\item[{a livello collegamento}] gateway, router, e bridge
\item[{a livello fisico}] gateway, router, bridge, e hub attivo
\end{description}

i dispositivi sono quindi
\begin{description}
\item[{hub passivo}] sta sotto l'hub attivo e agisce sotto il livello fisico, permette la continità del segnale tra cavi e reti distinti (internet mf)
\item[{hub attivo}] ripetitore, connette a livello fisico due parti della sottrete, rigenera il segnale
quando ci sono più ethernet con un hub in mezzo si comportano da lan, csma cd
\item[{bridge}] controlla gli indirizzi mac per l'accesso and company tramite la tablella del bridge, filtra e inoltra pacchetti, apparte il bridge può starci un
\item[{switch livello 2}] può operare per
\begin{description}
\item[{store and forward}] come un bridge, sicuro ma lento
\item[{cut through}] come router, veloce ma può provocare perdita di frame
\end{description}
lo switch non è visibile ai nodi che specificano solo gli indirizzi di rete
di base serve a filtrare ma funziona anche per inoltrare
importante l'autoapprendimento sulla tabella di inoltro
\item[{router}] processa pacchetti ricevuti dal livello di collegamento, li mette in un buffer, e dopo averli processati li inoltra sulla base di una tablella di routing, può operare in
\begin{description}
\item[{store and forward}] memorizza il pacchetto e lo inoltra
\item[{cut through}] inoltra i pacchetti anche senza la loro completa ricezione
\end{description}
\item[{gateway}] sinomimo di router, ma per livelli sopra la rete, tipo trasporto o appliczione
trasporta i pacchetti oltre la rete locale
\end{description}

\section{Wifi}
\label{sec:orgfd4c951}
Wireless fidelity
standard IEEE 802.11
si occupa solo delle specifiche a livello fisico e MAC
formata da una cella elementare detta BSS contentente uno o più BS (Base Station), questa coordina la rete.
nella wlan la BS è nota come Access Point (AP)

\subsection{MAC}
\label{sec:org51b0637}
livello mac comune a tutte le alternative disponibili per il livello fisso,
basato su CSMA CA\footnote{collisoin avoidance}, introdotta per la rete wireless
csma ca si basa sulla rilevazione di portante con l'aggiunta funzionalità di prevenire le collisoini
l'accesso al canale prevede che vi sia un ritardo detto ifs per l'invio del frame anche a canale libero
per ridurre l'intervallo di vulnerabilità del l'ifs è diviso in
\begin{description}
\item[{sifs}] fisso
\item[{aifs}] per dare priorità ai canali sensibili ai ritardi (es canali audio)
\item[{difs}] tempo di attesa di terminale una volta trovato il canale libero, ma prima di mandare un pacchetto
\item[{aifs}] per traffico a priorità più bassa, di "background"
\item[{eifs}] tempo di attesa rima che un terminale notifichi gli altri se riceve un frame divettoso
\end{description}
l'ordine degli intervalli stabilisce una priorità per l'accesso al canale

per accedere al canale si può fare
\begin{description}
\item[{alla cazzo di cane}] con \textbf{dcf}, se il canale è libero il terminale fa partire un contatore a decremento inizializzato a difs, quando il tempo scade, se è ancora libero, il terminale trasmette e resta in attesa dell'ack per un tempo sifs.
se non si riceve si assume collisione virtuale e si gestisce la collisione, stato di contesa.
entrati in contesa si prende una finestra di tempo, si sceglie in modo uniforme, e in pratica aloha

la csma ca presenta alcune criticità come il
\end{description}
\end{document}